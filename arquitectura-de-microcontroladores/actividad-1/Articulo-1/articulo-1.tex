
%----------------------------------------------------------------------------------------
%	PACKAGES AND DOCUMENT CONFIGURATIONS
%----------------------------------------------------------------------------------------

\documentclass{article}

\usepackage[version=3]{mhchem} % Package for chemical equation typesetting
\usepackage{siunitx} % Provides the \SI{}{} and \si{} command for typesetting SI units
\usepackage{graphicx} % Required for the inclusion of images
\usepackage{natbib} % Required to change bibliography style to APA
\usepackage{amsmath} % Required for some math elements 
\usepackage[utf8]{inputenc}
%\usepackage{natbib}
\setlength\parindent{0pt} % Removes all indentation from paragraphs

\renewcommand{\labelenumi}{\alph{enumi}.} % Make numbering in the enumerate environment by letter rather than number (e.g. section 6)

%\usepackage{times} % Uncomment to use the Times New Roman font

%----------------------------------------------------------------------------------------
%	DOCUMENT INFORMATION
%----------------------------------------------------------------------------------------

\title{Requisitos Industriales sobre la Evolución de la Arquitectura de Sistemas Embebidos} % Title


\author{Sergio Hernández Reyes} % Author name

\date{\today} % Date for the report

\begin{document}

\maketitle % Insert the title, author and date



% If you wish to include an abstract, uncomment the lines below
\renewcommand{\abstractname}{Abstracto}
\begin{abstract}
El siguiente es un resumen sobre el artículo \cite{ieee}.
\end{abstract}

%----------------------------------------------------------------------------------------
%	SECTION 1
%----------------------------------------------------------------------------------------
\section{Objectivo}
Poder tener una noción sobre la evolución de algún sistema embebido a travez del tiempo.



%----------------------------------------------------------------------------------------
%	SECTION 2
%----------------------------------------------------------------------------------------
\section{Desarrollo}
Se realiza una lectura del articulo completo resaltando las partes importantes del mismo.


%----------------------------------------------------------------------------------------
%	SECTION 3
%----------------------------------------------------------------------------------------
\section{Resultados}
{\bf Introducción}\\
El mantenimiento, actualización o reemplazo de piezas de software o hardware puede ser tedioso o imposible ya que un sistema como estos puede durar 30 años, te enfrentas a un problema en el que no sabes si reemplazar o actualizar debido a que se vuelve obsoleto.\\

Las necesidades pueden cambiar al ocupar más velocidad de red, más computo, o la introducción de una nueva tecnología que puede cambiar la arquitectura completa del sistema.\\
\\Las empresas se enfrentan a un proceso de evolución y esto nos lleva a buscar una manera de dejar a los sistemas lo más preparado posibles para esto, lo cual nos hace pensar e imaginar en posibles tecnologías que pudieran surgir.\\

\\{\bf Definición de problema}\\
Las empresas saben que al desarrollar un sistema se debe tomar en cuenta el echo de tener que actualizar sus componentes o su sistema entero de ser necesario, para ello se seleccionan las siguientes áreas que son las principales en la evolución de un sistema:\\

\begin{enumerate}
    \item {\bf Componentes obsoletos:} podría ser necesario modificar el hardware y esto nos llevaría a también modificar software que podría ser no compatible del todo con este, lo que nos lleva a que se tengan que hacer de nuevo las pruebas necesarias.
    \item {\bf Requisitos funcionales/no-funcionales:} al actualizar el hardware se puede obtener una ejecución de código mas rápida y por lo tanto una respuesta mas rápida que la esperada, pudiendo afectar si ya se tiene establecido un tiempo de respuesta.
Requisitos no-funcionales: rendimiento, seguridad, ciber seguridad.
    \item {\bf Prueba/Verificación:} Al actualizar un sistema viejo, deben realizarse las pruebas pertinentes para comprobar que el nuevo software/hardware funcionan en esta actualización sin afectar el rendimiento.
\end{enumerate}\\

\\{\bf Trabajos relacionados}\\
Hay muchos articulos de diferentes areas que intentan promover algun tipo de solucion a la actualizacion de sistemas obsoletos.



%----------------------------------------------------------------------------------------
%	SECTION 4
%----------------------------------------------------------------------------------------
\section{Conclusiones}
Podemos decir que en los sitemas embebidos nos enfrentaremos siempre a estos problemas, si tenemos un sistema embebido ya desarrollado, tarde o temprano tendremos que actualizarlo tanto por software o hardware. Así también si vamos a desarrollarlo desde cero debemos tomar esto en cuenta, pensando en dejar preparado de la mejor forma posible para que llegue el dia de actualizar o mejorar, no se nos complique o sea lo mas facil posible.

%\begin{figure}[h]
%\begin{center}
%\includegraphics[width=0.65\textwidth]{placeholder} % Include the image placeholder.png
%\caption{Figure caption.}
%\end{center}
%\end{figure}

%----------------------------------------------------------------------------------------
%	BIBLIOGRAPHY
%----------------------------------------------------------------------------------------

%\bibliographystyle{apalike}
\bibliographystyle{plain}
\renewcommand{\refname}{Referencias}
\bibliography{refs}
%----------------------------------------------------------------------------------------


\end{document}