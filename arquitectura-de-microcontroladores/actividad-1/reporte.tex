
%----------------------------------------------------------------------------------------
%	PACKAGES AND DOCUMENT CONFIGURATIONS
%----------------------------------------------------------------------------------------

\documentclass{article}

\usepackage[version=3]{mhchem} % Package for chemical equation typesetting
\usepackage{siunitx} % Provides the \SI{}{} and \si{} command for typesetting SI units
\usepackage{graphicx} % Required for the inclusion of images
\usepackage{natbib} % Required to change bibliography style to APA
\usepackage{amsmath} % Required for some math elements 
\usepackage[utf8]{inputenc}
%\usepackage{natbib}
\setlength\parindent{0pt} % Removes all indentation from paragraphs

\renewcommand{\labelenumi}{\alph{enumi}.} % Make numbering in the enumerate environment by letter rather than number (e.g. section 6)

%\usepackage{times} % Uncomment to use the Times New Roman font

%----------------------------------------------------------------------------------------
%	DOCUMENT INFORMATION
%----------------------------------------------------------------------------------------

\title{Arquitectura de Microcontroladores \\ Reporte de Práctica/Investigación \\ Introducción a la arquitectura de los microcontroladores y microprocesadores} % Title


\author{Sergio Hernández Reyes} % Author name

\date{\today} % Date for the report

\begin{document}

\maketitle % Insert the title, author and date



% If you wish to include an abstract, uncomment the lines below
 \begin{abstract}
 El siguiente artículo consiste en una investigación acerca de la arquitectura de los microcontroladores, para ello tomaremos 3 de diferentes marcas comparandolos en sus características de memoria.
 \end{abstract}

%----------------------------------------------------------------------------------------
%	SECTION 1
%----------------------------------------------------------------------------------------

\section{Objectivo}

Poder diferenciar y comparar entre distintos microcontroladores logrando así tener una visión acerca de las características de cada uno de ellos.


%----------------------------------------------------------------------------------------
%	SECTION 2
%----------------------------------------------------------------------------------------

\section{Desarrollo}
Se toman 3 microcontroladores de diferentes marcas para realizar esta comparación:

\begin{enumerate}
    \item {\bf PIC16F877A (Microchip)}
    \item {\bf MC68HC711E9 (Motorola)}
    \item {\bf Quark SE C1000 (Intel)}
\end{enumerate}


%----------------------------------------------------------------------------------------
%	SECTION 3
%----------------------------------------------------------------------------------------

\section{Resultados}
Con base a la información contenida en cada data sheet se obtuvieron los siguientes resultados:
\begin{enumerate}
    \item Para el microcontrolador {\bf PIC16F877A} se tiene una memoria SRAM de 368 bytes, memoria EEPROM de 256 bytes \cite{microship}.
    \item Para el microcontrolador {\bf MC68HC711E9} se tiene una memoria RAM de 512 bytes, memoria EPROM de 12 Kb, memoria EEPROM de 512 bytes \cite{motorola}.
    \item Para el microcontrolador {\bf Quark SE C1000} se tiene una memoria SRAM de 80 Kb, memoria ROM de 8Kb \cite{intel}.
\end{enumerate}

%----------------------------------------------------------------------------------------
%	SECTION 4
%----------------------------------------------------------------------------------------

\section{Conclusiones}
Con esto podemos ver que hay diferentes arquitecturas de diferentes marcas, cada empresa proporciona diferentes microcontroladores variando en cada uno características o capacidades, ya sera cuestión del programador elegir el que se adecúe más a las necesidades tomando en cuenta costos, rendimiento y características en general.

%\begin{figure}[h]
%\begin{center}
%\includegraphics[width=0.65\textwidth]{placeholder} % Include the image placeholder.png
%\caption{Figure caption.}
%\end{center}
%\end{figure}

%----------------------------------------------------------------------------------------
%	BIBLIOGRAPHY
%----------------------------------------------------------------------------------------

%\bibliographystyle{apalike}
\bibliographystyle{plain}
\renewcommand{\refname}{Referencias}
\bibliography{refs}
%----------------------------------------------------------------------------------------


\end{document}